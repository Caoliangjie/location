\documentclass{article}

\usepackage{ctex}

\usepackage{multicol}

\usepackage[top=1in, bottom=1in, left=1.25in, right=1.25in]{geometry}

\usepackage{lscape}

\author{Qi Zhao}

\date{April 17,2018}

\title{The upchuck wagon}


\begin{document}


\maketitle
\par There is a hope come from someone who boosts the idea of self-driving cars. Because someone hope they will be able to do other things, such as reading, working on the laptop or having a nap, when riding in such a vehicle. But it seems impossible according to a study published in 2014 by the University of Michigan.
\par Help is at hand. Motion sickness is caused by a conflict between signals arriving in the brain from the inner ear. The selfsame authors have just been awarded patent for a device that could act as a countermeasure against the malady. And the idea they come up with is to arrange for an array of small lights, which would be controlled by various motion sensors and blink on and off in a way that is designed to mimic the velocity, rolling, pitching and other movements of a vehicle. For the wearer of such kit, the effect would be to provide a visual response that corresponds to the movements the inner ear is detecting.
\par The prototype is being made, in the future, and is used by passengers who feel sick in cars with a driver at the wheel. Maybe it might also help with other forms of motion sickness, such as airsickness or seasickness. The technology will give people a more comfortable experience.


\end{document}
