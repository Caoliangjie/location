\documentclass{article}

\usepackage{ctex}

\usepackage{multicol}

\usepackage[top=1in, bottom=1in, left=1.25in, right=1.25in]{geometry}

\usepackage{lscape}

\usepackage{graphicx}

\usepackage[colorlinks=true]{hyperref}

\usepackage{subfigure}

\author{Qi Zhao}

\date{May 3,2018}

\title{GrAIt expectation}
\newcommand{\upcite}[1]{\textsuperscript{\textsuperscript{\cite{#1}}}}

\begin{document}


\maketitle
\par Artificial intelligence\cite{Rich1985Artificial} is spreading beyond the technology sector, with big consequences for companies, workers and consumers. For example, Ping An, a Chinese insurance company, thinks it can spot dishonesty, which lets customers apply for loans through its app. And it monitors around 50 tiny facial expressions to determine whether they are telling the truth.
\par AI will change more than borrowers�� bank balances and sort through job applications and pick the best candidates. Instead of relying on gut instinct and rough estimates, cleverer and speedier AI-powered predictions promise to make businesses much more efficient. AI and machine learning\cite{Mitchell2003Machine} involve computers crunching vast quantities of data to find patterns and make predictions\cite{Armstrong2012AI} without being explicitly programmed to do so.
\begin{figure}[htbp]
\centering
\includegraphics[width=0.5\textwidth]{AI.jpg}
\caption{Global activity in AI and machine learning}
\label{1}
\end{figure}
\par One of AI��s main effects will be a dramatic drop in the cost of making predictions. Most of today��s leading tech firms, such as Google and Amazon in the West and Alibaba and Baidu in China, would not be as big and successful without AI for product recommendations, targeted advertising and forecasting demand. Meanwhile, the talent of AI is thin of ground. And companies have been buying up promising young tech firms (see Fig.{\color{red}1}). The path ahead is exhilarating but perilous.
\bibliographystyle{ieeetr}
\bibliography{4}

\end{document}
