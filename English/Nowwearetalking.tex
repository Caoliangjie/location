\documentclass{article}

\usepackage{ctex}

\usepackage{multicol}

\usepackage[top=1in, bottom=1in, left=1.25in, right=1.25in]{geometry}

\usepackage{lscape}

\usepackage{graphicx}

\usepackage[colorlinks=true]{hyperref}

\usepackage{subfigure}

\author{Qi Zhao}

\date{May 7,2018}

\title{Now we��re talking}
\newcommand{\upcite}[1]{\textsuperscript{\textsuperscript{\cite{#1}}}}

\begin{document}


\maketitle
\par Voice technology\cite{Wikipedia2011Voice} is making computers less daunting and more accessible. Using voice computing\cite{Beritelli2006A} is just like casting a spell(see Fig.{\color{red}1}): say a few words into the air, and a nearby device can grant your wish. And more companies think dictating e-mails and text message now works reliable enough. This is a huge shift. Simple thought it may seem, voice has the power to transform computing, by providing a natural means of interaction. But Voice will not wholly replace other forms of input and output.
\begin{figure}[htbp]
\centering
\includegraphics[width=0.5\textwidth]{talking.jpg}
\caption{Voice interaction}
\label{1}
\end{figure}
\par However, to reach its full potential, the technology requires further breakthroughs-and a resolution of the tricky questions it raises around the trade-off between convenience and privacy. Many voice-driven devices\cite{Schimmer2002VOICE} are always listening, waiting to be activated. Some people are already concerned about the implications of internet-connected microphones listening in every room and from every smartphone. Even if such issues remain unresolved, consumers will adopt voice computing as always.
\bibliographystyle{ieeetr}
\bibliography{6}

\end{document}
