\documentclass{article}

\usepackage{ctex}

\usepackage{multicol}

\usepackage[top=1in, bottom=1in, left=1.25in, right=1.25in]{geometry}

\usepackage{lscape}

\usepackage{graphicx}

\usepackage[colorlinks=true]{hyperref}

\usepackage{subfigure}

\author{Qi Zhao}

\date{May 13,2018}

\title{Google wants to cure your smartphone addiction}
\newcommand{\upcite}[1]{\textsuperscript{\textsuperscript{\cite{#1}}}}

\begin{document}


\maketitle
\par Google has some ideas to make your phone a little more humane(see Fig.{\color{red}1}). It introduced features for its coming Android P operating system designed to help consumers and parents curb smartphone addiction and detox from our beloved glowing screens.
\begin{figure}[htbp]
\centering
\includegraphics[width=0.5\textwidth]{Google.jpg}
\caption{Sameer Samat speaks at the Google}
\label{1}
\end{figure}
\par The new Android features, which probably won��t come as an update to your phone for months, are focused on making it easier to activate no-distraction modes\cite{Elbialy2008Effects} and get feedback about how we��re using our phones. Among the highlights:
\par ��Wind down�� mode �� When it��s getting time for you to go to bed, your phone will now fade to gray, which will make apps become a lot less interesting without color.
\par Turn your phone face down to ��shush�� �� Flipping your phone onto its face will turn on ��do not disturb�� mode, which silences calls, buzzes and other visual notifications.
\par ��Time spent�� dashboard ��You can also set time limits for apps, after which their app icons will appear gray on your screen (but still work).
\par Watch your kids �� An app called Family Link will allow parents to control their kids�� devices.
\par The changes are welcome, but there��s more the tech companies could do, especially given how much data they collect about us. Android update is the part of a broader new digital well-being initiative that��s just beginning.
\bibliographystyle{ieeetr}
\bibliography{9}

\end{document}
