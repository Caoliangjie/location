\documentclass{article}

\usepackage{ctex}

\usepackage{multicol}

\usepackage[top=1in, bottom=1in, left=1.25in, right=1.25in]{geometry}

\usepackage{lscape}

\author{Qi Zhao}

\date{April 21,2018}

\title{Taming the titans}


\begin{document}


\maketitle
\par The early signs are already visible that Google, Facebook and Amazon are increasingly dominant. Not long ago, as the billions rolled in, so did the plaudits: Google, Facebook, Amazon and others were making the world a better place. Nowadays, these companies are accused of being BAADD��big, anti-competitive, addictive and destructive to democracy. So regulators fine them, politicians grill them and one-time backers warn of their power to cause harm. Because big tech platforms, particularly Facebook, Google and Amazon, do indeed raise a worry about fair competition, and they often benefit from legal exemptions.
\par Facebook and Google are rarely held responsible for what users do on them; and for years most American buyers on Amazon did not pay sales tax. Nor do the titans simply compete in a market. Increasingly, they are the market itself, providing the infrastructure (or ��platforms��) for much of the digital economy. Many of their services appear to be free, but users ��pay�� for them by giving away their data. They hold sway over the media industry the world��s largest pool of personal data, but also its biggest ��social graph�� ��the list of its members and how they are connected.
\par If this trend runs its course, consumers will suffer as the tech industry becomes less vibrant. And less money will go into startups, most good ideas will be bought up by the titans and, one way or another, the profits will be captured by the giants.


\end{document}
