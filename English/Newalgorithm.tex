\documentclass{article}

\usepackage{ctex}

\usepackage{multicol}

\usepackage[top=1in, bottom=1in, left=1.25in, right=1.25in]{geometry}

\usepackage{lscape}

\usepackage{graphicx}

\usepackage[colorlinks=true]{hyperref}

\usepackage{amsmath}

\usepackage{subfigure}

\author{Qi Zhao}

\date{May 21,2018}

\title{New algorithm repairs corrupted digital images in one step}
\newcommand{\upcite}[1]{\textsuperscript{\textsuperscript{\cite{#1}}}}

\begin{document}


\maketitle
\par From phone camera snapshots\cite{Schwaebel2014Photoscopy} to lifesaving medical scans, digital images play an important role in the way humans communicate information. But digital images are subject to a range of imperfections such as blurriness, grainy noise, missing pixels and color corruption.
\par A group has designed a new algorithm that incorporates artificial neural networks to simultaneously apply a wide range of fixes to corrupted digital images. The research team tested their algorithm by taking high-quality, uncorrupted images, purposely introducing severe degradations, then using the algorithm to repair the damage (see Fig.{\color{red}1}).
\begin{figure}[htbp]
\centering
\includegraphics[width=0.5\textwidth]{test}
\caption{ The research team artificially degraded a stock image, deliberately introducing blur, noise and other imperfections. BOTTOM: The research team's new image repair algorithm automatically returned the image to near-original quality.}
\label{1}
\end{figure}

\par Artificial neural networks\cite{Downing1999Evolving} are a type of artificial intelligence algorithm inspired by the structure of the human brain. They can assemble patterns of behavior based on input data, in a process that resembles the way a human brain learns new information. And the team has designed a new algorithm that incorporates artificial neural networks to simultaneously apply a wide range of fixes to corrupted digital images. The researchers tested their algorithm by taking high-quality, uncorrupted images, purposely introducing severe degradations, then using the algorithm to repair the damage.
\bibliographystyle{ieeetr}
\bibliography{13}

\end{document}
