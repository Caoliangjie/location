\documentclass{article}

\usepackage{ctex}

\usepackage{multicol}

\usepackage[top=1in, bottom=1in, left=1.25in, right=1.25in]{geometry}

\usepackage{lscape}

\author{Zhaoqi}

\date{April 10,2018}

\title{Affective picture processing: The late positive potential is modulated by motivational relevance}


\begin{document}


\maketitle
\par Recent there are some studies have shown that the late positive component of the event-related-potential is enhanced for emotional pictures, presented in an oddball paradigm, evaluated as distant from an established affective context.Current research shows that pleasant or unpleasant images can lead to selectively pay more attention to processes, which are reflected in the continuing positive ERP potential. This effect seems to depend on the motivational significance of these stimuli. To verify that the subject was attending to the pictures, an evaluation task was introduced. After picture offset, the subject was asked to evaluate each picture as either pleasant, neutral, or unpleasant using a three-way response button.Although the present study is not an exhaustive, parametric attempt to compare these two paradigms, the data in-dictate clearly that evocation of differential LPPs is not dependent on stimulus duration, content grouping, or stimulus presentation rate.
\par From the paper,we can recognize the differences be-tween affective pictures were seen to depend importantly on the oddball status of the target affective stimulus.


\end{document}
