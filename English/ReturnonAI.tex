\documentclass{article}

\usepackage{ctex}

\usepackage{multicol}

\usepackage[top=1in, bottom=1in, left=1.25in, right=1.25in]{geometry}

\usepackage{lscape}

\usepackage{graphicx}

\usepackage[colorlinks=true]{hyperref}

\usepackage{subfigure}

\author{Qi Zhao}

\date{May 11,2018}

\title{Return on AI}
\newcommand{\upcite}[1]{\textsuperscript{\textsuperscript{\cite{#1}}}}

\begin{document}


\maketitle
\par AI\cite{Bengio2009Learning} has already changed some activities, including parts of finance like fraud prevention, but not yet fund management and stock-picking. Machine learning\cite{Mitchell2003Machine}, a subset of AI that excels at finding patterns and making predictions using reams of data, looks like an ideal tool for the business. The reason by co-founder of Sentient Technologies is an AI startup with a hedge-fund arm and left to their own devices, machine-learning techniques are prone to ��overfit��, ie, to finding peculiar patterns in the specific data they are trained on that do not hold up in the wider world. This is especially true of financial data because of their comparative paucity. Many thinks this turns them into a ��pure math problem��. The idea is that this avoids biases creeping into models.
\begin{figure}[htbp]
\centering
\includegraphics[width=0.5\textwidth]{return.jpg}
\caption{AI-driven hedge funds need human brains}
\label{1}
\end{figure}
\par The fund uses machine learning not just to crunch data and come up with strategies. The classification system\cite{Macnamee2007Classification} that gauges the relative merits of these strategies is itself run by machine learning. But humans do the actual trading, following the algorithm��s instructions.
\bibliographystyle{ieeetr}
\bibliography{8}

\end{document}
