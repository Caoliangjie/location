\documentclass{article}

\usepackage{ctex}

\usepackage{multicol}

\usepackage[top=1in, bottom=1in, left=1.25in, right=1.25in]{geometry}

\usepackage{lscape}

\author{Qi Zhao}

\date{April 19,2018}

\title{Controversial apps for kids make cosmetic surgery into a game}


\begin{document}


\maketitle
\par A game that focuses on a petition calling for Apple, Google and Amazon, which is called ��Beauty Clinic Plastic Surgery�� --- one of many game apps. Because the petition demands by making cosmetic surgery apps available for download, Apple, Google and Amazon are allowing companies to stoke and profit from the insecurities of children. Although the petition in the United States gathered more than 119,000 signatures during the year-long campaign, the Apple spokesman Tom shows they do not want nor allow these types of apps on the store. And they have rules in place against these apps and do not offer them on the App Store. There are several problems with games that promote cosmetic surgery as part of a makeover routine. For one, in the game they promote the idea that someone's body is a ��thing to be fixed.�� The Worse is that the surgery element is treated so lightly, where this equates makeup and hair changes with serious procedures.
\par This is not a problem caused by technology, but companies have a responsibility to ensure their platforms send the right message. at the very least, the companies should do a better job of warning potential consumers about the content, which should be token the healthy growth of children into consideration.

\end{document}
