\documentclass{article}

\usepackage{ctex}

\usepackage{multicol}

\usepackage[top=1in, bottom=1in, left=1.25in, right=1.25in]{geometry}

\usepackage{lscape}

\usepackage{graphicx}

\usepackage[colorlinks=true]{hyperref}

\usepackage{subfigure}

\author{Qi Zhao}

\date{May 17,2018}

\title{Money mountains}
\newcommand{\upcite}[1]{\textsuperscript{\textsuperscript{\cite{#1}}}}

\begin{document}


\maketitle
\par We must take a moment to admire and fear for the ascent of America��s big-five tech firms. Apple, Alphabet, Microsoft, Amazon and Facebook have recently become the five most valuable listed companies in the world, you can see Fig. {\color{red}1}. And they are worth more than any five firms in history, you can get it in table {\color{red}1}.
\begin{figure}[htbp]
\centering
\includegraphics[width=0.5\textwidth]{money.jpg}
\caption{the five most valuable listed companies.}
\label{1}
\end{figure}
\begin{table}[!htbp]
  \centering
 \begin{tabular}{|c|c|c|c|c|c|c|}
    \hline
    year& 2007 & 2009 & 2011 & 2013 & 2015 & 2017\\
    \hline
    Combined market capitalisation/trn dollars & 0.55 & 0.32 & 0.78 & 1.21 & 1.85 & 2.45\\
    \hline
  \end{tabular}
  \caption{\textbf{American"formidable five" tech firms.}}
  \label{Table1}
  \end{table}
\par Elevated tech valuations used to be a sign of hysteria\cite{Beyer1984Neuro}. Today��s investors believe they are making an ice-cold judgment that these firms are the dominant oligopolies of the 21st century and will extract a vast, rising, flow of profits. Billions of users are tied into these firms�� social-media networks, digital assistants\cite{Navarrete1999In}, operating systems and cloud-computing platforms. But There is one gnawing doubt, however: the formidable five��s cash-rich balance-sheets,you can see Fig.{\color{red}2}, which are built as if they expect a crisis, not to dominate the world.
\begin{figure}[htbp]
\centering
\includegraphics[width=0.5\textwidth]{money1.jpg}
\caption{net cash.}
\label{2}
\end{figure}
\par Old-economy oligopolists finance themselves largely with debt, which is cheap but inflexible, and return most of the cash they make to shareholders. Yet, oddly, the biggest tech firms have the opposite approach. Together they have 330bn dollars of net cash (cash less debt), a ratio of twice their gross cashflow. The cash cushion is far larger than is needed to absorb shocks, such as a financial crash or a hacking attack.
\bibliographystyle{ieeetr}
\bibliography{11}

\end{document}
